\documentclass[10pt,letterpaper]{article}

% to do (5/16)

%% put more sparkles around writing OED in a program. Why is this clever and useful?

%% introduce max Entropy prior early (could also be interpolation between max ent and the predictive)
%%%% results of sequence prediction (formerly, "randomness") w.r.t. maxEnt vs. predictive (discuss why we think maxEnt is giving a better prediction here)

%% Figure 3 histogram: vertical line with AIG
%% create n_subjects analysis figure (a la FIgure 3, line plot) for sequence prediction... bootstrap up to n=100 to observe crossover

% if you need to pass options to natbib, use, e.g.:
% \PassOptionsToPackage{numbers, compress}{natbib}
% before loading nips_2016
%
% to avoid loading the natbib package, add option nonatbib:
\usepackage{cogsci}
\usepackage{pslatex}
\usepackage[natbibapa]{apacite}

%\usepackage{nips_2016}

% to compile a camera-ready version, add the [final] option, e.g.:
% \usepackage[final]{nips_2016}

% \usepackage[utf8]{inputenc} % allow utf-8 input
% \usepackage[T1]{fontenc}    % use 8-bit T1 fonts
%\usepackage{hyperref}       % hyperlinks
\usepackage{url}            % simple URL typesetting
\usepackage{booktabs}       % professional-quality tables
\usepackage{amsfonts}       % blackboard math symbols
\usepackage{nicefrac}       % compact symbols for 1/2, etc.
% \usepackage{microtype}      % microtypography
\usepackage{amsmath}
\usepackage{mathtools}
\usepackage{fancyvrb}
\usepackage{multirow}
\usepackage{color}
\usepackage{textcomp}
\usepackage[scaled=0.8]{inconsolata}


% HT http://tex.stackexchange.com/a/151987/41154
\DeclarePairedDelimiterX{\infdivx}[2]{(}{)}{%
  #1\;\delimsize\|\;#2%
}
\newcommand{\dkl}{D_\mathrm{KL}\infdivx}

\usepackage{listings}
\definecolor{lightgray}{rgb}{.9,.9,.9}
\definecolor{darkgray}{rgb}{.4,.4,.4}
\definecolor{purple}{rgb}{0.65, 0.12, 0.82}
\definecolor{orange}{rgb}{1,0.5,0}

\definecolor{Red}{RGB}{255,0,0}
\newcommand{\red}[1]{\textcolor{Red}{#1}}
\definecolor{Green}{RGB}{10,200,100}
\definecolor{Blue}{RGB}{10,100,200}
\newcommand{\ndg}[1]{\textcolor{Green}{[ndg: #1]}}
\newcommand{\mht}[1]{\textcolor{Blue}{[mht: #1]}}
\newcommand{\lou}[1]{\textcolor{orange}{[lou: #1]}}

% casual outlining font
\newcommand{\cas}[1]{ \textsf{\color{darkgray} \scriptsize #1} }

\lstdefinelanguage{JavaScript}{
  keywords={typeof, new, true, false, catch, function, return, null, catch, switch, var, if, in, while, do, else, case, break},
  keywordstyle=\color{blue}\bfseries,
  ndkeywords={class, export, boolean, throw, implements, import, this},
  ndkeywordstyle=\color{darkgray}\bfseries,
  identifierstyle=\color{black},
  sensitive=false,
  comment=[l]{//},
  morecomment=[s]{/*}{*/},
  commentstyle=\color{purple}\ttfamily,
  stringstyle=\color{red}\ttfamily,
  morestring=[b]',
  morestring=[b]"
}

\lstset{
   language=JavaScript,
   backgroundcolor=\color{white},
   extendedchars=true,
   basicstyle=\footnotesize\ttfamily,
   showstringspaces=false,
   showspaces=false,
   numbers=none,
   numberstyle=\footnotesize,
   numbersep=9pt,
   tabsize=2,
   breaklines=true,
   showtabs=false,
   captionpos=b
}



\usepackage[ruled,vlined]{algorithm2e}

\newcommand{\ud}{\,\mathrm{d}}
\DeclareMathOperator*{\argmax}{arg\,max}


\title{\texttt{webppl-oed}: A practical optimal experiment design system}

% The \author macro works with any number of authors. There are two
% commands used to separate the names and addresses of multiple
% authors: \And and \AND.
%
% Using \And between authors leaves it to LaTeX to determine where to
% break the lines. Using \AND forces a line break at that point. So,
% if LaTeX puts 3 of 4 authors names on the first line, and the last
% on the second line, try using \AND instead of \And before the third
% author name.

\author{
  Long Ouyang*, Michael Henry Tessler*, Daniel Ly, Noah D. Goodman\\
  Department of Psychology\\
  Stanford University\\
  Stanford, CA 94305 \\
  \texttt{\{louyang, mtessler, lydaniel, ngoodman\}@stanford.edu}\\
}

\begin{document}
% \nipsfinalcopy is no longer used

\maketitle

\begin{abstract}

An essential part of cognitive science is designing experiments that distinguish competing models.
This requires patience and ingenuity---there is often a large space of possible experiments one could run but only a small subset that might yield informative results.
But we need not comb this space by hand---if we use formal models and explicitly declare the space of experiments, we can automate the search for good experiments, looking for those with high \emph{expected information gain}.
Here, we present an automated system for experiment design called \texttt{webppl-oed}.
In our system, users simply declare their models and experiment space; in return, they receive a list of experiments ranked by their expected information value.
We demonstrate our system in two case studies, where we use it to design experiments in studies of sequence prediction and categorization.
%, but the framework is general to any domain where probabilistic models are used to represent theories.
We find strong empirical validation that our automatically designed experiments were indeed optimal.
%We conclude by discussing a number of interesting questions for future research.


\end{abstract}

%\mht{I think in practice ``the space of possible experiments'' is the same thing as an ``empirical paradigm''. This terminology may be useful in our exposition.}

\section{Introduction}
%\lou{start off more generally, don't specialize too much for psychology.}
%\ndg{agree: first talk about OED in general, and why we should do OED within a PPL setup, then talk about psych as a target domain.}
Cognitive scientists often design experiments to test competing computational models.
Good experiments are ones where the models make different predictions, but there are typically many possible experiments one could run.
As such, designing experiments where models sufficiently diverge is driven in large part by intuition, rather than by systematic search of the experiment space.
This intuition may be biased in a number of ways, such as towards experiments that show qualitative differences between models even when more informative quantitative differences may exist.

In principle, there is a better way---if we formally declare the space of models and space of experiments, optimal experiment design (OED) allows us to automate the search for good experiments (i.e., ones that strongly update our beliefs about a scientific question).
However, while the mathematical foundations of OED are fairly straightforward \citep{Lindley1956}, it has not enjoyed widespread use in practice.
Some OED systems are too specialized for general use and the more general systems require too much statistical and computational know-how to be widely adopted (e.g., users must supply their own objective function and derive a solution algorithm for it).
In this work, we describe an automated system that is both general and practical---the user writes the competing models and space of possible experiments in a common language and a set of potentially informative experiments is then computed with no further input from the user.

% a key obstacle has hindered its use in practice: a complex calculation is required to formulate the objective for each new set of models and experiments, which has brought out a pluralism of idiosyncratic approaches \cite{Chaloner1995}.
% Adapting the general formulation to one's specific problem can be sufficiently difficult and time-consuming to deter working scientists from using OED.


%This is a crucial step in making a practical OED system: once you use a PPL to formalize your hypotheses, OED for free.

We first describe our framework in general terms and then apply it in two case studies.
% Cognitive science is a good target domain for OED: hypotheses can often be expressed as mathematical models, rapid experiment iteration is possible and beneficial, and there is a large community ready to use (but not necessarily develop) sophisticated tools.
%Psychological experiments also have certain challenging features: human participants give noisy responses, experimental results are sensitive to the size of one's sample, and computational models often do not make direct predictions about experimental data, instead requiring \emph{linking functions} to convert model output into predictions about data.
%Our system naturally addresses these concerns.
First, we consider the problem of distinguishing three toy models of human sequence prediction.
Second, we go beyond toy models and analyze a classic paper \citep{medin78:pr} on human category learning that compared two models using a hand-designed experiment.
Our OED system discovers experiments that are several times more effective than the original in an information-theoretic sense.
Our work opens a number of rich areas for future development, which we explore in the discussion.

%We conclude by highlighting the generality of the approach and areas of future work.
%\ndg{all the parts are here. needs smoothing. make it clearer that using PPL to represent hypotheses is a key step in making the system practical.}

%    \begin{itemize}
%        \item It is difficult to discriminate models of psychological processes
%        \item Experiments are expensive
%        \item We present a general, turn-key approach to design experiments that best disambiguate competing models using a Bayesian framework
%        \item This technique is not directly related to Bayesian models of cognition. It can be used on any (formal / probabilistic) model, including Bayesian models of cognition
%        \item Despite the previous attempts in this field, there are a number of pragmatic issues that make it difficult to readily apply OED techniques for psychology, including:
%        \begin{itemize}
%            \item A variety of proposed optimization criteria, which puts the burden on researchers to have sufficient expertise to select the appropriate approach
%            \item A lack of an established pipeline, requiring researchers to develop a language to formalize psychological models and write an OED optimization engine
%            \item A lack of analysis in dealing with practical experimental concerns such as:
%                \begin{itemize}
%                    \item Noisy responses from participants
%                    \item The ideal number of participants for a study
%                    \item The ambiguity of linking functions of dependent measures
%                \end{itemize}
%        \end{itemize}
%    \end{itemize}


\section{Experiment design framework}
\label{s:bayes}
We begin with a concrete example before giving formal details.
Imagine that we are studying how people predict values for sequence data (e.g., flips of a possibly-trick coin).
We want to compare two models: $m_{\text{fair}}$, in which people believe the coin is unbiased, and $m_{\text{bias}}$, in which people believe the coin has a bias that is unknown (expressed as a uniform prior on the unit interval) but that can be learned from data.
%People would see a sequence of outcomes, and be asked to predict the next outcome.
\emph{A priori}, we do not believe either model more than the other; in Bayesian terms, we have a uniform prior on the models.
We wish to update this belief distribution by conducting an experiment where we show people four flips of the same coin and ask them to predict what will happen on the next flip.
There are 16 possible experiments (all combinations of \lstinline{H} and \lstinline{T} for 4 flips) and $2^n$ possible outcomes---predictions of heads or tails for each of $n$ human participants.
Each model makes a prediction about how people would respond after seeing some particular sequence of flips.
In other words, a model defines a probability distribution on $\{0,1\}^n$ conditional on the experiment $x$.
For convenience, we write our models in terms of a what a single person would do and assume that all people respond according to the same model, i.e., participant responses are i.i.d.\footnote{We use this simple linking function throughout this paper but our approach handles arbitrary linking functions (e.g., hierarchical models with subject-wise parameters).}

How informative would running the experiment \lstinline{HHTT} be?
$m_{\text{fair}}$ and $m_{\text{bias}}$ are identical here (heads and tails are equally likely)---if we ran our experiment with a single participant, neither experimental result would update our beliefs about the models, so this is a poor experiment.
By contrast, the experiment \lstinline{HHHH} would be much more informative.
Under $m_{\text{fair}}$, $p(\texttt{H}) = \frac{1}{2}$ but under $m_{\text{bias}}$, $p(\texttt{H}) = \frac{5}{6}$.
In this case, \emph{either} experimental response would be informative.
If the participant predicted heads, this would favor $m_{\text{bias}}$ and if she predicted tails, this would favor $m_{\text{fair}}$.
Thus, \lstinline{HHHH} would be a better experiment for disambiguating the models.
The goal of OED is to automate this kind of reasoning.

Now, we provide formal details.
We wish to compare a set of models $M$ in terms of how well they account for empirical phenomena.
A model $m$ is a conditional distribution $P_m(Y \mid X)$ representing the likelihood of empirical results $y$ for different possible experiments $x$.
We adopt a Bayesian model comparison approach---we begin with a prior on models $P(M)$ and aim to conduct an experiment $x^*$ that maximally updates this distribution, providing as much information as possible.
That is, we wish to maximize the \emph{information gain} $\dkl{ P(M \mid X = x^*) }{ P(M) }$.
%$$x^{*} = \argmax_{x} \dkl{ P(M \mid X = x) }{ P(M) }.$$
\emph{A priori}, we do not know what the result of any particular experiment will be, so we compute the \emph{expected information gain} by marginalizing over the possible results $y$:
\begin{align}
  x^{*} &= \argmax_{x} {\mathbb E}_{p(y ; x)} \dkl{ P(M \mid X = x, Y = y) }{ P(M) }  \label{eq:oed}
\end{align}
where $p(y ; x)$ is the probability of observing result $y$ for experiment $x$.
If we have reason to believe that $M$ contains the true model of the data, then a suitable choice for $p(y ; x)$ is the predictive distribution implied by the models $p(y ; x) = {\mathbb E}_{p(m)} p_m(y \mid x)$.
If, however, we think $M$ may not contain the true model, then an uninformative prior $p(y ; x) \propto 1$ may be more appropriate.

It is worth pointing out that we commit to a Bayesian perspective only for the model comparison problem; the models themselves need not be Bayesian---they only need to define a probability distribution specifying predictions for different experiments.
Additionally, our method does not rely on exactly finding the global maximum $x^*$.
OED can still be useful, as it can it find good experiments that we would not have discovered by hand.

Finally, it is worth acknowledging that not all computational models yield probability distributions on experiment results.
For instance, models may make deterministic predictions about what should happen in an experiment.
In some cases, however, it is possible to make additional assumptions that yield probabilities.
For example, a deterministic prediction can be interpreted as a mean, median, or modal tendency of subjects' responses and that actual responses are spread around this value.
Similarly, deterministic predictions about a single subject's responses can be made probabilistic by assuming that subjects respond with some noise level.

\subsection{Writing models as probabilistic programs}

Because models are probability distributions, we have the user express their models in a programming language where probability distributions and operations on them are first-class objects.
In particular, we use WebPPL (\url{webppl.org}), a small but feature-rich probabilistic programming language embedded in Javascript \citep{dippl}.

WebPPL supports sampling from a number of primitive probability distributions, which can be combined in various ways. For example, we can define a distribution that adds Gaussian noise to a Binomial random variable:

\begin{lstlisting}[mathescape, label={code:forward-model-simple}]
var g = function() {
  var x = sample(Binomial({n: 4, p: 0.5}))
  var y = sample(Gaussian({mu: 0, sigma: 1}))
  x + y // function returns its last expression
}
Infer(g)
\end{lstlisting}
Here, \lstinline{g} is a function that defines a sampling procedure for our compound distribution.
\lstinline{g} implicitly represents a probability distribution, but to reify this into an actual distribution, we must perform inference \texttt{Infer(g, options)}.
This separation between \emph{what} we wish to compute from \emph{how} we try to compute it is useful when writing larger, more complex models.
Note that in the above snippet, and throughout, we omit the \texttt{options} object, which specifies which inference algorithm to use.\footnote{\,WebPPL currently provides these inference algorithms: MCMC (MH, HMC), SMC, enumeration, and variational inference.}

WebPPL also supports expressing \emph{conditional} probability distributions.
For instance, in the model above, we might be interested in what values of $x$ and $y$ could lead their sum to be greater than 2:

\begin{lstlisting}[mathescape, label={code:forward-model-complex}]
var g = function() {
  var x = sample(Binomial({n: 4, p: 0.5}))
  var y = sample(Gaussian({mu: 0, sigma: 1}))
  condition(x + y >= 2)
  [x, y]
}
Infer(g)
\end{lstlisting}
Here, \texttt{condition} rejects any states where $x + y <$ 2.

Given a set of competing models and some details about the space of possible experiments and results, \texttt{webppl-oed} searches for experiments that have a high expected information gain.
The software is available online at \url{https://censored}. %--- \url{https://github.com/mhtess/webppl-oed}.
We next illustrate our system by applying it to distinguish toy models of sequence prediction.
%\ndg{check nips anonymity requirements...}

\section{Case study 1: Sequence prediction}
\label{s:tutorial}

Human judgments about sequences are surprisingly systematic and nonuniform across equally likely outcomes -- for example, we might strongly believe the next coin flip in the sequence \lstinline{HHTTHHTT} will be \lstinline{H}, whereas we might be unsure for the sequence \lstinline{THHTHTHT}.
There are many hypotheses one might have about what underlies human intuitions about such sequences \citep{goodfellow38:jep, falk81:pme, Griffiths2004_nips}.
Here, we consider three simple models of people's beliefs: (a) \emph{Fair coin}: people assume the coin is fair, (b) \emph{Bias coin}: people believe the coin has some unknown bias (i.e., the probability of a \lstinline{H} outcome) that they can learn from data, (c) \emph{Markov coin}: people believe the coin has some probability of transitioning between spans of \lstinline{H} and \lstinline{T} outcomes, also learnable from the data.
As in our earlier example, we consider an experimental setup where participants see four flips of the same coin and must predict the next flip.

\subsection{Formalization}

The model space $M$ is $\{m_{\text{fair}}, m_{\text{bias}}, m_{\text{markov}}\}$.
For now, we assume that the experiment will collect data from just a single participant, so the experiment space $X$ is the Cartesian product $\{1\} \times \{\texttt{H}, \texttt{T}\}^4$ representing the fixed sample size of 1 and sequence space.\footnote{\,Our notion of ``experiment'' is quite general, including traditional components like stimulus properties (e.g., coin sequence) as well as other components like dependent measure and sample size.}
Finally, $Y$ is the response set $\{\texttt{H}, \texttt{T}\}^1$.

Under $m_{\text{fair}}$, people assume that the coin always has an equal probability of coming up heads or tails:
\begin{lstlisting}[upquote=true]
var fairCoin = function(seq) {
  Infer(function(){ flip(0.5) })
}
\end{lstlisting}
Here, \texttt{flip(0.5)} is shorthand for \texttt{sample(Bernoulli({p:0.5}))}.
Note the type signature of this model---it takes as input an experiment and returns a distribution on possible results of that experiment.

Under $m_{\text{bias}}$, people assume that the coin has some unknown bias, learn it from past observations\footnote{\,The line that uses \texttt{condition} constrains likely values of the coin weight---this mechanism is used to represent learning in Bayesian models of cognition. For more, see the Learning as Conditional Inference chapter of the online textbook \url{http://probmods.org}.}, and use it to predict the next flip:
\begin{lstlisting}[upquote=true]
var coinWeights = [0.01, 0.10, 0.20, ..., 0.90, 0.99];
var biasCoin = function(seq) {
  Infer(function(){
    var w = uniformDraw(coinWeights),
        coinFlip = function(){ flip(w) },
        sampledSeq = repeat(seq.length, coinFlip)
    condition(arrayEquals(seq,sampledSeq))
    coinFlip()
  })
}
\end{lstlisting}
Under $m_{\text{markov}}$, people assume that the flips are generated by a Markov process with transition probability \texttt{p}, which is learned from past observations:

\begin{lstlisting}
var markovCoin = function(seq) {
  Infer(function(){
    var p = uniformDraw(coinWeights)
    var sampleOne = function(lastFlip) {
      if (flip(p)) {
        !lastFlip
      } else {
        lastFlip
      }
    }
	  var sampleSeq = function(seqSoFar, n) {
      if (n == 0) {
        seqSoFar
      } else {
        var nextFlip = sampleOne(last(seqSoFar))
        var nextSeq = append(seqSoFar, nextFlip)
        sampleSequence(nextSeq, n - 1)
      }
    }
	  var sampledSeq = sampleSeq([flip(0.5)],
                               seq.length - 1)
    condition(arrayEquals(seq, sampledSeq))
    sampleOne(last(sampledSeq))
  })
}
\end{lstlisting}

\subsection{Predictions of optimal experiment design}

\newsavebox{\LstBox}

\begin{lrbox}{\LstBox}
\begin{lstlisting}
var groupify = function(model) {
  var groupified = function(x) {
    var sequence = x.sequence, n = x.n;
    var singleModel = model(sequence);
    var p = Math.exp(singleModel.score(true))
    Binomial({n: n, p: p})
  }
  groupified
}
\end{lstlisting}
\end{lrbox}

Using an uninformative prior for $p(y; x)$, we ran OED for three different model comparisons: fair--bias, bias--Markov, and fair--bias--Markov and planned to collect data from 20 participants (rather than 1).\footnote{\,Our models are of a single subject but we lift each single-participant model into a model of group responses using an i.i.d. linking function that we call \texttt{groupify}:
\usebox{\LstBox}\\
Here, \texttt{singleModel.score(true)} returns the log-probability of the value \texttt{true} under the \texttt{singleModel} distribution.}
We run \texttt{OED} by writing:
\begin{lstlisting}
var n = 20,
    fairGroup = groupify(fairCoin),
    biasGroup = groupify(biasCoin)
OED({
  M: function() { uniformDraw([fairGroup, biasGroup]) },
  X: function() {
    {n: n, seq: uniformDraw(["HHHH",...,"TTTT"])}
  },
  Y: function(x) { randomInteger(n + 1) }
})
\end{lstlisting}
We define a uniform prior on models \texttt{M}, an experiment space \texttt{X} with a fixed number of subjects and all valid coin sequences, and a result space \texttt{Y}, which is the uninformative prior over the number of \texttt{H} responses.
The results of different model comparisons are below:

\begin{figure}[h]
 \includegraphics[width=\columnwidth]{img/coin_eig_n20_ignorance.pdf}
  \caption{Results for sequence prediction model comparisons}
  \label{fig:run-coin}
\end{figure}

Consider the fair--bias comparison (Fig.~\ref{fig:run-coin}b, left).
Several experiments have 0 information gain (e.g., \lstinline{HTHT})---the models make exactly the same predictions in this case (albeit for different reasons), so the experiment has no distinguishing power.
The best experiments are \lstinline{HHHH} and \lstinline{TTTT}.
This is intuitive---the bias model infers a strongly biased coin and makes a strong prediction, while the fair coin model is unaffected by past observations.

In the bias--Markov comparison (Fig.~\ref{fig:run-coin}b, middle), the best and worst experiments actually reverse.
Now, \lstinline{HHHH} and \lstinline{TTTT} are the least informative (because, as before, the models make similar predictions here), whereas \lstinline{HTHT} and \lstinline{THTH} are the most informative.
This makes sense---the bias model learns a weight of 0.5 and so assigns equal probability of heads and tails to the next flip, whereas the Markov model learns that the transition probability is high and assigns high probability to the opposite of the most recent flip (\lstinline{T} for \lstinline{THTH} and \lstinline{H} for \lstinline{HTHT}).

\begin{figure}[t]
            (a) \\ \includegraphics[width=1.0\columnwidth]{img/coin_predictions.pdf}\\ (b) \\ \includegraphics[width=1.0\columnwidth]{img/coin_eig_3way_nsubj_wlegend.pdf}
  \caption{(a) Model predictions for top three experiments (HHHH, HHHT, HTHT) in the full comparison (b) Expected information gain for these experiments versus sample size.}
  \label{fig:coin_preds}
\end{figure}

In the full fair--bias--Markov comparison (Fig.~\ref{fig:run-coin}b, right), the worst experiments (e.g., \lstinline{TTHH}) are again cases where all models make similar predictions.
The best experiments are \lstinline{TTTT} and \lstinline{HHHH}, a result that is non-obvious because we are comparing three models rather than two.
The best experiment \lstinline{HHHH} is very good at separating the fair model from the other two models, while still predicting a difference between bias and Markov (Fig.~\ref{fig:coin_preds}a, right).
The second best experiment, \lstinline{HHHT}, better distinguishes the bias model from the Markov model as it predicts a qualitative difference (Fig.~\ref{fig:coin_preds}a, middle), but this comes at the cost of less expected information gain overall.
An automated design tool is especially useful in these settings, where human intuition would likely favor the qualitative over the quantitative difference.

Finally, expected information gain of an experiment varies as a function of sample size (Fig.~\ref{fig:coin_preds}b).
This function is non-linear and, crucially, the rank ordering of experiments can change.
For the the full model comparison, the experiments \lstinline{HTHT} and \lstinline{HHHT} switch places after 12 participants.
This is particularly relevant when three models are being compared, as small quantitative differences between two models may amplify as the sample size grows.
%In our example here, the optimal experiment with 1 participant is the same as with 30 participants.


\subsection{Empirical validation}
We validated our system by collecting human judgements for all 16 experiments and comparing expected information gain with the actual information gain from the empirical results.
We randomly assigned 351 participants to an experiment (all of the 16 experiments were completed by $\geq$20 unique participants).
Participants pressed a key to sequentially reveal the sequence of 4 flips and then predicted the next coin flip (either heads or tails).

\begin{figure}[t]
 \includegraphics[width=1.0\columnwidth]{img/coin_eig_aig_scatter_noText.pdf}
 \caption{Actual vs. Expected Information Gain for each experiment}
  \label{fig:aig_vs_eig}
\end{figure}

For each experiment $x$ and result $y$, we computed the expected information gain from running our empirical sample of participants\footnote{\,N's are uneven due to randomization. We use the empirical N's for EIG in comparisons to AIG.} and compared this to the actual information gain, $\dkl{P(M \mid Y = y, X = x)}{P(M)}$, for the three model comparison scenarios.
Figure \ref{fig:aig_vs_eig} shows that expected information gain is a reliable predictor of the empirical value of an experiment (minimum $r$ = 0.857). This indicates that the OED tool could be relied on to automatically choose good experiments for this case study.


\section{Case study 2: Category learning}

Here, we explore a more complex and realistic space of models and experiments.
In particular, we analyze a classic paper on the psychology of categorization by \cite{medin78:pr} that aimed to distinguish two competing models of category learning -- the \emph{exemplar model} and the \emph{prototype model}.
Using intuition, Medin and Schaffer (MS) designed an experiment (often referred to as the ``5-4 experiment'') where the models made diverging predictions and found that the results supported the exemplar model.
Subsequently, many other authors followed their lead, replicating and using this experiment to test other competing models.
Here, we ask: how good was the MS 5-4 experiment?
Could they have run an experiment that would have distinguished the two models with less data?
%was more information-theoretically efficient?
%Using our OED framework, we find that there are many superior experiments that Medin and Schaffer could have designed but did not.

%\footnote{Our work here is an exercise in counterfactual history; the Medin and Schaffer models are not state of the art. We chose the Medin and Schaffer research (rather than newer work) as an object of study because it commits to a clear set of competing models and a clear set of possible experiments.}



\subsection{Models}

Both the exemplar and prototype models are classifiers that map inputs (objects represented as a vector of Boolean features) to a probability distribution on the categorization response (a label: A or B).
The exemplar model assumes people store information about every instance of the category they have observed; categorizing an object is thus a function of the object's similarity to all of the examples of category A versus the similarity to all of B's examples.
By contrast, the prototype model assumes that people store a measure of central tendency for each category---a prototype.
Categorization of an object is thus a function of its similarity to the A prototype versus its similarity to the B prototype.
For space, we omit these model implementations but refer interested readers to the source code available online.

\subsection{Experiments}

Participants first learn about the category structures in a training phase where they perform supervised learning of a subset of the objects and are then tested on this learning in a test phase.
During training, participants see a subset of the objects presented one at a time and must label each object.
Initially, they can only guess at the labels, but they receive feedback so that they can eventually learn the category assignments.
After reaching a learning criterion, they complete the test phase, where they label all the objects (training set and the held out test set) without feedback.

MS used visual stimuli that varied on 4 binary dimensions (color: \emph{red} vs. \emph{green}, shape: \emph{triangle} vs. \emph{circle}, size: \emph{small} vs. \emph{large}, and count: \emph{1} vs. \emph{2}).
For technical reasons, they considered only experiments that (1) have linearly separable decision boundaries, (2) contain 5 A's and 4 B's in the training set, and (3) have the modal A object \lstinline{1111} and the modal B object \lstinline{0000}.
There are, up to permutation, 933 experiments that satisfy these constraints.

\subsection{Predictions of optimal experimental design}

Using the predictive prior for $p(y; x)$, we computed the expected information gain for all 933 experiments and found that the best experiment (for a single participant) had an expected information gain of 0.08 nats, whereas the MS 5-4 experiment had an expected information gain of only 0.03 nats.
Thus, the optimal experiment is expected to be 2.5 times more informative than the MS experiment.
Indeed, the MS experiment is near the bottom third of all experiments (Fig.~\ref{fig:dist}a).

Why is the MS experiment less effective?
One reason is that Medin and Schaffer prioritized experiments that predict a qualitative categorization difference (i.e., when one model predicts that an object is an A while the other predicts it is a B).
The experiment they designed indeed predicts a qualitative difference for one object but this difference has a small magnitude and comes at the expense of little information gain from the remaining objects.
The optimal experiment is better able to quantitatively disambiguate the models by maximizing the information from all the objects simultaneously.

\begin{figure}[t]
(a) \\
\includegraphics[width=1.0\columnwidth]{img/category-eig-dist.pdf}\\
(b) \\
\includegraphics[width=1.0\columnwidth]{img/category-aig-curve.pdf}
\caption{(a) Distribution of expected information gain for all possible category learning experiments for a single participant. MS has low expected information gain. (b) Actual information gain versus number of experimental participants included in analysis (error bars are 95\% bootstrapped confidence intervals). MS requires three times as many participants to achieve maximum actual information gain.}
\label{fig:dist}
\end{figure}

\subsection{Empirical validation}

To validate our expected information gain calculations, we ran the MS 5-4 and the optimal experiment with 60 participants each.
Figure~\ref{fig:dist}b shows that the optimal experiment we found for a single participant is indeed better than the MS experiment ($n$=1, blue greater than red).
For $n$=1, the mean actual information gain for the optimal experiment is 0.15, whereas it is 0.026 for the MS experiment.
This 5-fold difference in informativity is even greater than the 2.5-fold difference  predicted by expected information gain.
In addition, by incrementally introducing more data, we observe that both experiments achieve maximal actual information gain but the optimal experiment takes only 10 participants to asymptote to this maximum whereas the MS experiment takes around 30.
Thus, the optimal experiment provides the same amount of information for a third of the experimental cost.

% and illustrates how automated experiment design can outperform human intuition. In particular, this case study demonstrates the efficacy of OED in psychology for discrete and non-ordinal experiment spaces, large combinatoric experiment spaces, and parametric model classes.

\section{Related work}

The basic intuition behind OED---to find experiments that maximize some measure of expected informativeness---has been independently discovered in a number of fields, including physics \citep{vanDenBerg2003}, chemistry \citep{Huan2010}, biology \citep{Vanlier2012, Liepe2013}, psychology \citep{Myung2009}, statistics \citep{Lindley1956}, and machine learning \citep{Golovin2010}.

Previous work, however, has either been too narrow for general use or required too much statistical and computational expertise.
For example, \cite{Liepe2013} devised a method for finding experiments that optimize information gain for parameters of biomolecular models (ODEs with Gaussian noise).
\cite{Myung2009} described a more general optimization method but this requires users to select their own utility function for the value of an experiment and implement inference on their own.
For example, they compared six memory retention models using Fisher Information Approximation as a utility function and performed inference using a custom annealed SMC algorithm.
Such ``bring-your-own'' requirements impose a significant burden on users and are a real barrier to entry.

By contrast, our OED system is general and practical, which allows users to rapidly explore different spaces of models, experiments, and inference algorithms.
In addition, it naturally handles certain challenging features of cognitive science experiments: human participants give noisy responses, experimental results are sensitive to the size of one's sample, and computational models often do not make direct predictions about experimental data, instead requiring \emph{linking functions} to convert model output into predictions about data.
Finally, our work is the first to (1) demonstrate that expected information gain is a reliable predictor of actual information gain and to (2) characterize the cost benefits of OED.

\section{Conclusion}

Cognitive scientists aim to design experiments that yield informative results.
\texttt{webppl-oed} partially automates experiment design, searching for experiments that maximally update beliefs about the model distribution.
With our approach, the scientist writes her models as probabilistic programs, sketches a space of possible experiments, and hands these to OED for experiment selection.
We stress that our work \emph{complements} scientists; it does not replace them.
Our tool eliminates the need to manually comb large spaces for good experiments; we hope this will free scientists and engineers to work on the more interesting problems---devising empirical paradigms and building models.

Our approach suggests a number of interesting directions for future work.
First, OED can be computationally challenging, so there is still room for optimizing search algorithms.

%First, we chose KL divergence between posterior and prior as our informativeness measure because it is a well-known divergence function, but other choices (e.g., TV distance) might also be suitable.
Second, we have examined model comparison problems where there are a finite number of models.
We believe that our approach also works in (1) parameter learning settings where the goal is to conduct experiments that best update beliefs about continuous parameters of a model, and (2) model comparison problems where a finite number of models each have continuous parameters that are unknown and must be integrated over.

Second, we have restricted attention to ``one-shot'' experiments, but it would be interesting to extend our work to sequential settings (e.g., adaptive tests).
Adaptive testing can be formulated as a problem of information gain of \emph{sequences} of experiments, which produce dependent and non-identical responses.
Some preliminary work suggests that \texttt{webppl-oed} can be profitably extended to the adaptive setting.


%
%Fourth, we have ignored the cost of experiments, but it would be worth explicitly taking this into account.
%Our informativeness approach could be usefully integrated with for real-world applications (e.g., MRI studies of rare patient populations, expensive aerospace experiments).

% \bibliographystyle{ieeetr}
\bibliographystyle{apacite}
\bibliography{oed_cogsci_2018}

\end{document}
